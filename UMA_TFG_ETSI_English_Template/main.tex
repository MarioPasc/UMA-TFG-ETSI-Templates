% ======================================================================
%  UMA – ETSI Informática · Static Back Cover
%  example.tex (2025-07-02)
%  Author: Mario Pascual González <mpascual@uma.es>
% ======================================================================

% ======================================================================
% Nota:

% Gracias por usar mi plantilla para tu Trabajo Fin de Grado. Si tienes
% cualquier comentario para mejorarla, o quieres hacerlo tú mism@, puedes
% ponerte en contacto conmigo en mpascual@uma.es, o crear un pull request 
% en https://github.com/MarioPasc/UMA-TFG-ETSI-Templates 

% El proyecto puede tardar bastante en compilar por primera vez, es normal. 
% Una vez se hagan varias compilaciones, la caché de Overleaf se llenará y
% cargará más rápido. 
% Si quieres ahorrar tiempo de carga, puedes comentar las líneas 
%     \usepackage{sty_files/cotutor_cover-blueuma}
%     \usepackage{sty_files/cotutor_cover-whiteuma}
% Si no tienes un co-tutor asociado (y vas a usar \MakePlain{Blue/White}UMACover)
% o, si tienes un co-tutor asociado, comentar las líneas
%      \usepackage{sty_files/plain_cover-blueuma}
%      \usepackage{sty_files/plain_cover-whiteuma}
% Y vas a usar \Make{Blue/White}UMACover
% ======================================================================
\documentclass{report}

%───────── Plantilla, portadas y contraportada ─────────────────────────
\usepackage[malacitana]{template/sty_files/template-TFGenglish-uma}

\usepackage{template/sty_files/cotutor_cover-blueuma}
\usepackage{template/sty_files/cotutor_cover-whiteuma}

\usepackage{template/sty_files/plain_cover-blueuma}
\usepackage{template/sty_files/plain_cover-whiteuma}

\usepackage{template/sty_files/backcover-umaes}

%───────── Paquetes auxiliares ─────────────────────────────────────────
\usepackage{blindtext}
\usepackage{mwe}               % imágenes de prueba
\usepackage{csquotes}
\usepackage[
  backend=biber,
  style=numeric,
  sorting=nyt
]{biblatex}
\addbibresource{references.bib}

%======================================================================
\begin{document}

\frontmatter
%────────────────────── Blue front cover ───────────────────────────────
\MakeBlueUMACover{
  degree     = {Grado en Ingeniería Informática},
  mencion    = {Ingeniería del Software},
  titleES    = {Game of Life Acelerado por GPU para Performance Audiovisual en Tiempo Real},
  titleEN    = {GPU-Accelerated Game of Life for Real-Time Audiovisual Performance},
  author     = {John Doe},
  tutor      = {Dr.~John Doe},
  cotutor    = {Dra.~Jane Doe},
  dept       = {Departamento de Lenguajes y Ciencias de la Computación},
  cityDate   = {Málaga, \today},
  logoLeft   = {template/logos/NEG-uma-logo.png},
  logoRight  = {template/logos/NEG-etsi-logo.png}
}

%────────────────────── White front cover ──────────────────────────────
\MakeWhiteUMACover{
  degree      = {Grado en Ingeniería Informática},
  mencion     = {Ingeniería del Software},
  titleES     = {Game of Life Acelerado por GPU para Performance Audiovisual en Tiempo Real},
  titleEN     = {GPU-Accelerated Game of Life for Real-Time Audiovisual Performance},
  author      = {John Doe},
  tutor       = {Dr.~John Doe},
  cotutor     = {Dra.~Jane Doe},
  dept        = {Departamento de Lenguajes y Ciencias de la Computación},
  cityDate    = {MÁLAGA, \today},
  defenseDate = {septiembre de 2025},
  logoLeft    = {template/logos/POS-uma-logo.png},
  logoRight   = {template/logos/POS-etsi-logo.png}
}

%────────────────────── Resumen / Abstract ─────────────────────────────
\renewcommand{\abstractname}{Resumen}
\begin{abstract}


\medskip
\noindent\textbf{Palabras clave:} 
\end{abstract}

\renewcommand{\abstractname}{Abstract}
\begin{abstract}


\medskip
\noindent\textbf{Keywords:} 
\end{abstract}

%──────────────── Agradecimientos ──────────────────────────────────────
% Sección opcional, si no se quiere usar, borrar el entorno umaacknowledgments

% Descomentar si los vas a escribir en inglés
% \SetAcknowledgmentsName{Acknowledgments} 
\begin{umaacknowledgments}

\end{umaacknowledgments}

%────────────────────── Índice y listas ────────────────────────────────
\cleardoublepage
\tableofcontents
\cleardoublepage
\listoffigures
\cleardoublepage
\listoftables
\cleardoublepage

%────────────────────── Nomenclatura ───────────────────────────────────
\nomenclature{}{}
\printnomenclature
\cleardoublepage

\mainmatter
%────────────────────── Cuerpo ─────────────────────────────────────────
% https://en.wikipedia.org/wiki/Standing_on_the_shoulders_of_giants 



%────────────────────── Contraportada ──────────────────────────────────
\backmatter
\printbibliography[title={Bibliography}]
\cleardoublepage
\MakeUMABackCover
\end{document}
% ======================================================================
