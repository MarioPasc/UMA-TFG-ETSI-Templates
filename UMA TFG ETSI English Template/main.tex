% Tu TFG empieza aquí
% https://en.wikipedia.org/wiki/Standing_on_the_shoulders_of_giants 

\documentclass{report}

%───────── Template, covers & back cover ──────────────────────────────
\usepackage{sty_files/template-TFGespaniol-uma}      

\usepackage{sty_files/cotutor_cover-blueuma}
\usepackage{sty_files/cotutor_cover-whiteuma}

\usepackage{sty_files/plain_cover-blueuma}
\usepackage{sty_files/plain_cover-whiteuma}

\usepackage{sty_files/backcover-umaes}

%───────── Helper packages ───────────────────────
\usepackage{csquotes}
\usepackage[
  backend=biber,
  style=numeric,
  sorting=nyt
]{biblatex}
\addbibresource{references.bib}

%======================================================================
\begin{document}
\frontmatter
%────────────────────── Blue front cover ───────────────────────────────

\MakeBlueUMACover{ 
  degree     = {Grado en Ingeniería Informática},
  mencion    = {Tecnologías del Software}, % Puedes dejarlo en blanco
  titleES    = {Generación Procedural de Terrenos mediante Ruido Perlin y Erosión Hidráulica en Tiempo Real},
  titleEN    = {Procedural Terrain Generation via Perlin Noise and Real-Time Hydraulic Erosion},
  author     = {John Doe},
  tutor      = {Dr.\ John Doe},
  cotutor    = {Dra.\ Jane Doe},
  dept       = {Departamento de Lenguajes y Ciencias de la Computación},
  cityDate   = {Málaga, \today},
  logoLeft   = {logos/NEG-uma-logo.png},
  logoRight  = {logos/NEG-etsi-logo.png}
}

%────────────────────── White front cover ──────────────────────────────

\MakeWhiteUMACover{ 
  degree      = {Grado en Ingeniería Informática},
  mencion     = {Tecnologías del Software},
  titleES     = {Generación Procedural de Terrenos mediante Ruido Perlin y Erosión Hidráulica en Tiempo Real},
  titleEN     = {Procedural Terrain Generation via Perlin Noise and Real-Time Hydraulic Erosion},
  author      = {John Doe},
  tutor       = {Dr.\ John Doe},
  cotutor     = {Dra.\ Jane Doe},
  dept        = {Departamento de Lenguajes y Ciencias de la Computación},
  cityDate    = {MÁLAGA, \today},
  defenseDate = {septiembre de 2025},
  logoLeft   = {logos/POS-uma-logo.png},
  logoRight  = {logos/POS-etsi-logo.png}
}

%────────────────────── Abstract / Resumen ─────────────────────────────
\begin{abstract}

\medskip
\noindent\textbf{Keywords:} 
\end{abstract}

\renewcommand{\abstractname}{Resumen}
\begin{abstract}


\medskip
\noindent\textbf{Palabras clave:} 
\end{abstract}
%\addcontentsline{toc}{chapter}{Resumen}
\renewcommand{\abstractname}{Abstract}

% Para agradecimientos in inglés, descomentar la siguiente línea:
%\SetAcknowledgmentsName{Acknowledgments}

\begin{umaacknowledgments}
...  
\end{umaacknowledgments}


%────────────────────── Contents & lists ────────────────────────────────
\cleardoublepage
\tableofcontents
\cleardoublepage
\listoffigures
\cleardoublepage
\listoftables
\cleardoublepage

%────────────────────── Nomenclature ────────────────────────────────────
\nomenclature{$x,y$}{Horizontal coordinates in height-field grid}
\printnomenclature
\cleardoublepage

\mainmatter
%────────────────────── Chapter 1 — Introduction ────────────────────────
\chapter{Introduction}


%────────────────────── Back matter ────────────────────────────────────
\backmatter
\printbibliography
\cleardoublepage
\MakeUMABackCover
\end{document}
%======================================================================
