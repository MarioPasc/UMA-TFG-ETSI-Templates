% ======================================================================
%  UMA – ETSI Informática · Static Back Cover
%  example.tex (2025-07-02)
%  Author: Mario Pascual González <mpascual@uma.es>
% ======================================================================

% ======================================================================
% Nota:

% Gracias por usar mi plantilla para tu Trabajo Fin de Grado. Si tienes
% cualquier comentario para mejorarla, o quieres hacerlo tú mism@, puedes
% ponerte en contacto conmigo en mpascual@uma.es, o crear un pull request 
% en https://github.com/MarioPasc/UMA-TFG-ETSI-Templates 

% El proyecto puede tardar bastante en compilar por primera vez, es normal. 
% Una vez se hagan varias compilaciones, la caché de Overleaf se llenará y
% cargará más rápido. 
% Si quieres ahorrar tiempo de carga, puedes comentar las líneas 
%     \usepackage{sty_files/cotutor_cover-blueuma}
%     \usepackage{sty_files/cotutor_cover-whiteuma}
% Si no tienes un co-tutor asociado (y vas a usar \MakePlain{Blue/White}UMACover)
% o, si tienes un co-tutor asociado, comentar las líneas
%      \usepackage{sty_files/plain_cover-blueuma}
%      \usepackage{sty_files/plain_cover-whiteuma}
% Y vas a usar \Make{Blue/White}UMACover
% ======================================================================

\documentclass{report}

%───────── Template, covers & back cover ──────────────────────────────
\usepackage{sty_files/template-TFGenglish-uma}      

\usepackage{sty_files/cotutor_cover-blueuma}
\usepackage{sty_files/cotutor_cover-whiteuma}

\usepackage{sty_files/plain_cover-blueuma}
\usepackage{sty_files/plain_cover-whiteuma}

\usepackage{sty_files/backcover-umaes}

%───────── Helper packages ───────────────────────
\usepackage{blindtext}          
\usepackage{mwe}                
\usepackage{csquotes}
\usepackage[
  backend=biber,
  style=numeric,
  sorting=nyt
]{biblatex}
\addbibresource{references.bib}

%======================================================================
\begin{document}
\frontmatter
%────────────────────── Blue front cover ───────────────────────────────

% Nota:
% Si tu TFG tiene un Co-tutor asociado, usa el comando \MakeBlueUmaCover;
% Si tu TFG *NO* tiene un Co-tutor asociado, usa el comando \MakePlainBlueUmaCover

\MakeBlueUMACover{ % <--- \MakeBlueUmaCover o \MakePlainBlueUmaCover (sin cotutor)
  degree     = {Grado en Ingeniería Informática},
  mencion    = {Tecnologías del Software}, % Puedes dejarlo en blanco
  titleES    = {Generación Procedural de Terrenos mediante Ruido Perlin y Erosión Hidráulica en Tiempo Real},
  titleEN    = {Procedural Terrain Generation via Perlin Noise and Real-Time Hydraulic Erosion},
  author     = {John Doe},
  tutor      = {Dr.\ John Doe},
  cotutor    = {Dra.\ Jane Doe},
  dept       = {Departamento de Lenguajes y Ciencias de la Computación},
  cityDate   = {Málaga, \today},
  logoLeft   = {logos/NEG-uma-logo.png},
  logoRight  = {logos/NEG-etsi-logo.png}
}

%────────────────────── White front cover ──────────────────────────────

% Nota:
% Si tu TFG tiene un Co-tutor asociado, usa el comando \MakeWhiteUmaCover;
% Si tu TFG *NO* tiene un Co-tutor asociado, usa el comando \MakePlainWhiteUmaCover


\MakeWhiteUMACover{ % <--- \MakeWhiteUmaCover o \MakePlainWhiteUmaCover (sin cotutor)
  degree      = {Grado en Ingeniería Informática},
  mencion     = {Tecnologías del Software},
  titleES     = {Generación Procedural de Terrenos mediante Ruido Perlin y Erosión Hidráulica en Tiempo Real},
  titleEN     = {Procedural Terrain Generation via Perlin Noise and Real-Time Hydraulic Erosion},
  author      = {John Doe},
  tutor       = {Dr.\ John Doe},
  cotutor     = {Dra.\ Jane Doe},
  dept        = {Departamento de Lenguajes y Ciencias de la Computación},
  cityDate    = {MÁLAGA, \today},
  defenseDate = {septiembre de 2025},
  logoLeft   = {logos/POS-uma-logo.png},
  logoRight  = {logos/POS-etsi-logo.png}
}

%────────────────────── Abstract / Resumen ─────────────────────────────
\begin{abstract}
We propose a GPU-accelerated pipeline for synthesising large-scale, realistic
digital terrains.  An octaved Perlin-noise fractal provides the elevation
base-map; a real-time hydraulic-erosion simulation then sculpts river networks
and sediment layers.  The resulting meshes render at \SI{120}{fps} on consumer
hardware and achieve superior realism scores in a double-blind user study.

\medskip
\noindent\textbf{Keywords:} procedural generation, Perlin noise, hydraulic
erosion, GPU computing, computer graphics
\end{abstract}

\renewcommand{\abstractname}{Resumen}
\begin{abstract}
Se presenta una tubería acelerada por GPU para sintetizar terrenos digitales
a gran escala con realismo.  Un fractal de ruido Perlin octavado genera el
mapa base de elevación; a continuación, una simulación de erosión hidráulica
en tiempo real esculpe redes fluviales y capas de sedimento.  Las mallas
resultantes se representan a \SI{120}{fps} en hardware de consumo y alcanzan
puntuaciones de realismo superiores en un estudio de usuario a doble ciego.

\medskip
\noindent\textbf{Palabras clave:} generación procedural, ruido Perlin, erosión
hidráulica, computación GPU, gráficos por ordenador
\end{abstract}
%\addcontentsline{toc}{chapter}{Resumen}
\renewcommand{\abstractname}{Abstract}

%──────────────── Acknowledgments / Agradecimientos ────────────────────

% Para agradecimientos in inglés, descomentar la siguiente línea:
%\SetAcknowledgmentsName{Acknowledgments}

\begin{umaacknowledgments}
...  
\end{umaacknowledgments}


%────────────────────── Contents & lists ────────────────────────────────
\cleardoublepage
\tableofcontents
\cleardoublepage
\listoffigures
\cleardoublepage
\listoftables
\cleardoublepage

%────────────────────── Nomenclature ────────────────────────────────────
\nomenclature{$x,y$}{Horizontal coordinates in height-field grid}
\nomenclature{$h(x,y)$}{Terrain height at $(x,y)$}
\nomenclature{$\alpha$}{Persistence parameter of fractal noise}
\nomenclature{$d_t$}{Water depth at time step $t$}
\nomenclature{$s_t$}{Sediment carried at time step $t$}
\printnomenclature
\cleardoublepage

\mainmatter
%────────────────────── Chapter 1 — Introduction ────────────────────────
\chapter{Introduction}
Procedural terrain synthesis is a cornerstone of open-world games and virtual
environments.  Figure~\cref{fig:pipeline} outlines our two-stage pipeline.

\begin{figure}[ht]
  \centering
  \includegraphics[width=.8\linewidth]{example-image-a}
  \caption{Pipeline: Perlin-noise terrain seed followed by hydraulic erosion.}
  \label{fig:pipeline}
\end{figure}

%────────────────────── Chapter 2 — Theory ──────────────────────────────
\chapter{Theoretical Background}
\section{Noise Theory and Terrain Fractals}

\subsection{Perlin Noise}
Perlin noise \(N(x,y)\) is defined as the weighted sum of gradient dot
products.  A fractal terrain is obtained via the $L$-octave sum
\begin{equation}
  h(x,y) \;=\; \sum_{o=0}^{L-1} \alpha^{\,o} \,
  N\!\bigl(2^{o}x,\,2^{o}y\bigr),
  \label{eq:fractal}
\end{equation}
where \(\alpha\in(0,1)\) controls persistence.

\begin{definition}\label{def:fbm}
The mapping in \cref{eq:fractal} is called \emph{fractional Brownian motion
(fBm)} terrain.
\end{definition}

\begin{theorem}[Spectral slope of fBm]\label{thm:slope}
The power spectrum of \(h\) obeys \(P(f)\propto f^{-(2H+1)}\) where
\(H=-\log_2\alpha\) is the Hurst exponent.
\end{theorem}

\begin{proof}[Sketch]
Apply Parseval’s theorem to each octave and sum the resulting geometric series.
\end{proof}

\subsection{Hydraulic Erosion Model}
We adopt the particle-based algorithm of \cite{green2005implementing}.  Water depth
\(d_t\) and sediment \(s_t\) evolve through advection and deposition governed
by
\[
  d_{t+1} = d_t - \nabla\!\cdot(d_t\mathbf{v}_t)\,,\qquad
  s_{t+1} = s_t + \Delta E - \Delta D.
\]

%────────────────────── Chapter 3 — Methods ─────────────────────────────
\chapter{Methodology}
\section{GPU Implementation}

\begin{algorithm}
  \caption{Particle-based Hydraulic Erosion}\label{alg:erosion}
  \begin{algorithmic}[1]
    \Procedure{Erode}{$h$, $N_\text{drops}$}
      \For{$i \gets 1$ \textbf{to} $N_\text{drops}$}
        \State Spawn droplet at random $(x,y)$, init $d,s,v$
        \While{droplet \textsc{active}}
          \State $g \gets \nabla h$ \Comment{height gradient}
          \State $v \gets \textsc{UpdateVelocity}(v,g)$
          \State $(x,y) \gets (x,y) + v\Delta t$
          \State $(h,s,d) \gets \textsc{ErodeDeposit}(h,s,d)$
        \EndWhile
      \EndFor
    \EndProcedure
  \end{algorithmic}
\end{algorithm}

\section{Reference Implementation}
\Cref{lst:py-noise} shows a concise Python fBm synthesiser:

\begin{lstlisting}[language=Python,caption={Python: fBm Height-Field},label={lst:py-noise}]
import numpy as np

def fbm(nx: int, ny: int, octaves: int = 5, alpha: float = 0.5):
    """Return an (nx, ny) fractional Brownian motion height-field."""
    h = np.zeros((nx, ny))
    for o in range(octaves):
        freq = 2 ** o
        amp  = alpha ** o
        h += amp * perlin_noise(freq, nx, ny)  # user-provided
    return h
\end{lstlisting}

%────────────────────── Chapter 4 — Results ─────────────────────────────
\chapter{Results}
\section{Visual Quality}
User study results (\cref{tab:user}) show a clear preference for our terrain.

\begin{table}[ht]
  \centering
  \caption{Mean realism score (1–7 Likert, $n=32$)}
  \label{tab:user}
  \begin{tabular}{lS[table-format=1.1]S[table-format=1.1]}
    \toprule
    Method & {Mean} & {SD} \\
    \midrule
    Unreal Engine noise  & 4.8 & 1.2 \\
    Ours (noise+erosion) & 6.2 & 0.7 \\
    \bottomrule
  \end{tabular}
\end{table}

\section{Performance}
\Cref{fig:fps} plots frame-rates vs.\ terrain size.

\begin{figure}[ht]
  \centering
  \includegraphics[width=.7\linewidth]{example-image-b}
  \caption{Render frame-rate on RTX-3060.}
  \label{fig:fps}
\end{figure}

%────────────────────── Chapter 5 — Discussion ─────────────────────────
\chapter{Discussion}
\blindtext[2]

%────────────────────── Appendices ──────────────────────────────────────
\begin{umaappendices}
\section{Code Examples in Multiple Languages}

\subsection{C Example}
\begin{lstlisting}[language=C,caption={C: Linear Congruential RNG},label={lst:c-lcg}]
#include <stdint.h>
static uint32_t seed = 1;
uint32_t lcg_rand() {          /* 32-bit LCG */
    seed = 1664525u * seed + 1013904223u;
    return seed;
}
\end{lstlisting}

\subsection{C++ Example}
\begin{lstlisting}[language=C++,caption={C++17: Fast Perlin Noise},label={lst:cpp-perlin}]
#include <random>
double grad(int hash, double x, double y);          // prototype

double perlin(double x, double y) {
    const int X = int(x) & 255, Y = int(y) & 255;
    x -= int(x); y -= int(y);
    double u = fade(x), v = fade(y);                // 6t^5-15t^4+10t^3
    int A = p[X] + Y, AA = p[A], AB = p[A + 1];
    int B = p[X + 1] + Y, BA = p[B], BB = p[B + 1];
    return lerp(v,
        lerp(u, grad(p[AA], x, y),   grad(p[BA], x-1, y)),
        lerp(u, grad(p[AB], x, y-1), grad(p[BB], x-1, y-1)));
}
\end{lstlisting}

\subsection{Python Example}
See \cref{lst:py-noise} in Chapter~3.

\subsection{MATLAB Example}
\begin{lstlisting}[language=Matlab,caption={MATLAB: Gaussian Blur},label={lst:matlab-blur}]
function y = gaussianBlur(x, sigma)
    kSize = ceil(6*sigma);
    g = fspecial('gaussian', kSize, sigma);
    y = imfilter(x, g, 'replicate');
end
\end{lstlisting}

\subsection{R Example}
\begin{lstlisting}[language=R,caption={R: Elevation Histogram},label={lst:r-hist}]
elev <- as.vector(terrain_matrix)
hist(elev, breaks = 50, col = "steelblue",
     main = "Elevation distribution",
     xlab = "Height (m)")
\end{lstlisting}
\end{umaappendices}

%────────────────────── Back matter ────────────────────────────────────
\backmatter
\printbibliography
\cleardoublepage
\MakeUMABackCover
\end{document}
%======================================================================
