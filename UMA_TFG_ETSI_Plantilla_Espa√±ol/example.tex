% ======================================================================
%  UMA – ETSI Informática
%  example.tex (2025-07-18)
%  Author: Mario Pascual González <mpascual@uma.es>
% ======================================================================

% ======================================================================
% Nota:

% Gracias por usar mi plantilla para tu Trabajo Fin de Grado. Si tienes
% cualquier comentario para mejorarla, o quieres hacerlo tú mism@, puedes
% ponerte en contacto conmigo en mpascual@uma.es, o crear un pull request 
% en https://github.com/MarioPasc/UMA-TFG-ETSI-Templates 

% El proyecto puede tardar bastante en compilar por primera vez, es normal. 
% Una vez se hagan varias compilaciones, la caché de Overleaf se llenará y
% cargará más rápido. 
% Si quieres ahorrar tiempo de carga, puedes comentar las líneas 
%     \usepackage{sty_files/cotutor_cover-blueuma}
%     \usepackage{sty_files/cotutor_cover-whiteuma}
% Si no tienes un co-tutor asociado (y vas a usar \MakePlain{Blue/White}UMACover)
% o, si tienes un co-tutor asociado, comentar las líneas
%      \usepackage{sty_files/plain_cover-blueuma}
%      \usepackage{sty_files/plain_cover-whiteuma}
% Y vas a usar \Make{Blue/White}UMACover
% ======================================================================
\documentclass{report}

%───────── Plantilla, portadas y contraportada ─────────────────────────
\usepackage[malacitana]{template/sty_files/template-TFGenglish-uma}

\usepackage{template/sty_files/cotutor_cover-blueuma}
\usepackage{template/sty_files/cotutor_cover-whiteuma}
\usepackage{template/sty_files/plain_cover-blueuma}
\usepackage{template/sty_files/plain_cover-whiteuma}
\usepackage{template/sty_files/backcover-umaes}

%───────── Paquetes auxiliares ─────────────────────────────────────────
\usepackage{blindtext}
\usepackage{mwe}               % imágenes de prueba
\usepackage{csquotes}
\usepackage[
  backend=biber,
  style=numeric,
  sorting=nyt
]{biblatex}
\addbibresource{references.bib}

%======================================================================
\begin{document}

\frontmatter
%────────────────────── Blue front cover ───────────────────────────────

% Nota:
% Si tu TFG tiene un Co-tutor asociado, usa el comando \MakeBlueUmaCover;
% Si tu TFG *NO* tiene un Co-tutor asociado, usa el comando \MakePlainBlueUmaCover
\MakeBlueUMACover{
  degree     = {Grado en Ingeniería Informática},
  mencion    = {Ingeniería del Software},
  titleES    = {Game of Life Acelerado por GPU para Performance Audiovisual en Tiempo Real},
  titleEN    = {GPU-Accelerated Game of Life for Real-Time Audiovisual Performance},
  author     = {John Doe},
  tutor      = {Dr.~John Doe},
  cotutor    = {Dra.~Jane Doe},
  dept       = {Departamento de Lenguajes y Ciencias de la Computación},
  cityDate   = {Málaga, \today},
  logoLeft   = {template/logos/NEG-uma-logo.png},
  logoRight  = {template/logos/NEG-etsi-logo.png}
}

%────────────────────── White front cover ──────────────────────────────

% Nota:
% Si tu TFG tiene un Co-tutor asociado, usa el comando \MakeWhiteUmaCover;
% Si tu TFG *NO* tiene un Co-tutor asociado, usa el comando \MakePlainWhiteUmaCover
\MakeWhiteUMACover{
  degree      = {Grado en Ingeniería Informática},
  mencion     = {Ingeniería del Software},
  titleES     = {Game of Life Acelerado por GPU para Performance Audiovisual en Tiempo Real},
  titleEN     = {GPU-Accelerated Game of Life for Real-Time Audiovisual Performance},
  author      = {John Doe},
  tutor       = {Dr.~John Doe},
  cotutor     = {Dra.~Jane Doe},
  dept        = {Departamento de Lenguajes y Ciencias de la Computación},
  cityDate    = {MÁLAGA, \today},
  defenseDate = {septiembre de 2025},
  logoLeft    = {template/logos/POS-uma-logo.png},
  logoRight   = {template/logos/POS-etsi-logo.png}
}

%────────────────────── Resumen / Abstract ─────────────────────────────
\renewcommand{\abstractname}{Resumen}
\begin{abstract}
Presentamos un motor del \emph{Game of Life} totalmente paralelizado en la GPU
que genera, de forma simultánea, texturas de vídeo 4-K a \SI{144}{fps} y flujos
MIDI sincronizados para síntesis musical.  La emulación se implementa como
\emph{compute shader} en \texttt{WGSL} y se ejecuta sobre WebGPU, facilitando
la portabilidad a navegadores modernos.  Se compara el rendimiento con
implementaciones clásicas en CPU y CUDA y se ilustran usos en
\emph{live-coding} audiovisual.

\medskip
\noindent\textbf{Palabras clave:} autómatas celulares, WebGPU, síntesis
procedural, música generativa, gráficos en tiempo real
\end{abstract}

\renewcommand{\abstractname}{Abstract}
\begin{abstract}
We present a fully GPU-parallelised \emph{Game of Life} engine that renders
4-K video textures at \SI{144}{fps} while streaming synchronised MIDI notes for
generative music.  The automaton is expressed as a compute shader in
\texttt{WGSL} and runs on WebGPU, providing browser-level portability.
Performance is benchmarked against CPU and CUDA baselines, and live-coding
use-cases are demonstrated.

\medskip
\noindent\textbf{Keywords:} cellular automata, WebGPU, procedural synthesis,
generative music, real-time graphics
\end{abstract}

%──────────────── Agradecimientos ──────────────────────────────────────
\begin{umaacknowledgments}
A mis compañer@s de \emph{live coding}, que inspiran cada iteración, y al
equipo del estándar WebGPU por abrir la puerta al cómputo en navegadores.
\end{umaacknowledgments}

%────────────────────── Índice y listas ────────────────────────────────
\cleardoublepage
\tableofcontents
\cleardoublepage
\listoffigures
\cleardoublepage
\listoftables
\cleardoublepage

%────────────────────── Nomenclatura ───────────────────────────────────
\nomenclature{$n$}{Longitud lateral de la retícula $n\times n$}
\nomenclature{$S_t$}{Matriz binaria de estados en el paso $t$}
\nomenclature{$\mathcal N(i,j)$}{Vecindad de Von Neumann de la celda $(i,j)$}
\nomenclature{$\delta$}{Delta de Kronecker}
\nomenclature{$f_{\text{MIDI}}$}{Frecuencia de generación de notas (Hz)}
\printnomenclature
\cleardoublepage

\mainmatter
%────────────────────── Capítulo 1 — Introducción ──────────────────────
\chapter{Introducción}
Los autómatas celulares (AC) son sistemas dinámicos discretos capaces de
comportamientos emergentes complejos.  El \emph{Game of Life} (GoL) de Conway
es quizá el AC más icónico y se ha empleado en visualizaciones artísticas,
generadores pseudoaleatorios y plataformas de computación no convencionales.
La Figura~\cref{fig:overview} esquematiza la tubería audiovisual propuesta.

\begin{figure}[ht]
  \centering
  \includegraphics[width=.8\linewidth]{example-image-a}
  \caption{Visión general del sistema: la simulación en GPU alimenta una unidad
           de texturas y un codificador MIDI para VJ en tiempo real.}
  \label{fig:overview}
\end{figure}

%────────────────────── Capítulo 2 — Fundamentos ───────────────────────
\chapter{Fundamentos}
\section{Regla de actualización}
Sea \(S_t\in\{0,1\}^{n\times n}\) la retícula en el instante~\(t\).  
La transición \(\Phi:\,S_t\mapsto S_{t+1}\) se define, elemento a elemento, por
\begin{equation}
  S_{t+1}(i,j) \;=\;
  \begin{cases}
    1 &\text{si } S_t(i,j)=1 \wedge N_8(i,j)\in\{2,3\},\\
    1 &\text{si } S_t(i,j)=0 \wedge N_8(i,j)=3,\\
    0 &\text{en otro caso},
  \end{cases}
  \label{eq:gol}
\end{equation}
donde \(N_8(i,j)=\sum_{(u,v)\in\mathcal N_8(i,j)}S_t(u,v)\) cuenta los vecinos
de Moore.

\begin{definition}[Período del \emph{glider}]
Un patrón \(P\subseteq S\) tiene \emph{período~\(k\)} si
\(\Phi^k(P)=P\) salvo traslación.  El \emph{glider} cumple \(k=4\).
\end{definition}

\begin{theorem}[Completitud de Turing {\cite{sakata1990extension}}]
El \emph{Game of Life} puede simular una máquina de Turing universal.
\end{theorem}

\section{Asignación MIDI}
Las activaciones celulares se codifican en números de nota MIDI mediante
\(
  n_{ij} = 60 + (i + 7j)\bmod 12
\),
creando una textura cromática que evoluciona con \(S_t\).

%────────────────────── Capítulo 3 — Metodología ───────────────────────
\chapter{Metodología}
\section{Diseño del \emph{kernel} GPU}
\begin{algorithm}
  \caption{Actualización del Game of Life en GPU mediante \emph{tiles}}
  \label{alg:gpu-life}
  \begin{algorithmic}[1]
    \Procedure{LifeStepKernel}{$S_t$}
      \State $\textbf{tile} \gets \textsc{SharedMem}(TILE\_SIZE)$
      \State \textsc{LoadTile}$(S_t,\textbf{tile})$
      \Statex
      \ForAll{$(i,j)\in\textbf{tile}$ \textbf{en paralelo}}
        \State $n \gets \sum_{(u,v)\in\mathcal N_8} \textbf{tile}[u,v]$
        \State $\textbf{out}[i,j] \gets 
               ( \textbf{tile}[i,j] \wedge (n\!=\!2 \vee n\!=\!3))
               \vee (\lnot\textbf{tile}[i,j] \wedge n\!=\!3)$
      \EndFor
      \State \textsc{Commit}$(\textbf{out}\rightarrow S_{t+1})$
    \EndProcedure
  \end{algorithmic}
\end{algorithm}

\section{Implementación de referencia (Python + NumPy)}
\begin{lstlisting}[language=Python,
  caption={Python: Paso de Game of Life en WebGPU},label={lst:py-life}]
import wgpu, numpy as np

@wgpu.compute
def life_kernel(s_in: wgpu.Buffer[(n, n), "u8"],
                s_out: wgpu.Buffer[(n, n), "u8"]):
    x, y = wgpu.thread_id.x, wgpu.thread_id.y
    n = sum(s_in[x+dx, y+dy] for (dx, dy) in NEIGHBORS)
    alive = s_in[x, y]
    s_out[x, y] = (alive and (n in (2, 3))) or (not alive and n == 3)
\end{lstlisting}

%────────────────────── Capítulo 4 — Resultados ────────────────────────
\chapter{Resultados}
\section{Rendimiento}
Las tasas de \emph{frame} en función del tamaño de la rejilla se muestran en la
Figura~\cref{fig:fps-performance}.

\begin{figure}[ht]
  \centering
  \includegraphics[width=.7\linewidth]{example-image-b}
  \caption{FPS en una RTX-3060 (media de 1 000 pasos).}
  \label{fig:fps-performance}
\end{figure}

\begin{table}[ht]
  \centering
  \caption{Comparación de rendimiento CPU vs. GPU}
  \label{tab:perf}
  \begin{tabular}{lS[table-format=4.0]S[table-format=2.1]}
    \toprule
    Backend & {Celdas / ms} & {Energía (W)} \\
    \midrule
    CPU O3           & 8\,120 & 65.2 \\
    CUDA (980 Ti)    & 94\,300 & 187.4 \\
    WebGPU (RTX-3060)& 83\,600 & 170.1 \\
    \bottomrule
  \end{tabular}
\end{table}

\section{Estudio de usuarios}
Los participantes (\(n=24\)) prefirieron nuestra salida audiovisual frente a un
shader de referencia con una puntuación media de \(6.4\pm0.8\) en escala Likert.

%────────────────────── Capítulo 5 — Discusión ──────────────────────────
\chapter{Discusión}
\blindtext

%────────────────────── Apéndices ──────────────────────────────────────
\begin{umaappendices}

\umaappendix{Ejemplos de código en múltiples lenguajes}

\section{Ejemplo en C}
\begin{lstlisting}[language=C,
  caption={C: Paso ingenuo de Game of Life en CPU},label={lst:c-life}]
void life_step(uint8_t *s, uint8_t *o, int n) {
  for (int i=1;i<n-1;++i)
    for (int j=1;j<n-1;++j) {
      int k = i*n + j;
      int n8 = s[k-1-n]+s[k-n]+s[k+1-n]+
               s[k-1  ]        +s[k+1  ]+
               s[k-1+n]+s[k+n]+s[k+1+n];
      o[k] = (s[k] && (n8==2 || n8==3)) || (!s[k] && n8==3);
    }
}
\end{lstlisting}

\section{Ejemplo en C++}
\begin{lstlisting}[language=C++,
  caption={C++17: Lanzamiento de un shader WGSL},label={lst:cpp-dispatch}]
#include <webgpu/webgpu.hpp>
void runLife(wgpu::Device dev, wgpu::Buffer in, wgpu::Buffer out, int n) {
    wgpu::ComputePipeline pipe = makePipeline(dev, "life.wgsl");
    wgpu::CommandEncoder enc = dev.CreateCommandEncoder();
    wgpu::ComputePassEncoder pass = enc.BeginComputePass();
    pass.SetPipeline(pipe);
    pass.SetBindGroup(0, makeBindGroup(dev, in, out));
    pass.DispatchWorkgroups(n/16, n/16);
    pass.End();
    dev.GetQueue().Submit({enc.Finish()});
}
\end{lstlisting}

\section{Ejemplo en MATLAB}
\begin{lstlisting}[language=Matlab,
  caption={MATLAB: Life mediante convolución},label={lst:matlab-life}]
function S = lifeStep(S)
    N = conv2(S, ones(3), 'same') - S;
    S = (S & (N==2 | N==3)) | (~S & N==3);
end
\end{lstlisting}

\section{Ejemplo en R}
\begin{lstlisting}[language=R,
  caption={R: Histograma de estados},label={lst:r-hist}]
hist(as.vector(S), breaks = 2,
     main = "Vivas vs. Muertas", xlab = "Estado")
\end{lstlisting}

\umaappendix{Otros recursos}
\section{Trabajos futuros}
\blindtext[1]
\end{umaappendices}

%────────────────────── Contraportada ──────────────────────────────────
\backmatter
\printbibliography[title={Bibliografía}]
\cleardoublepage
\MakeUMABackCover
\end{document}
% ======================================================================
